\documentclass[12pt]{report}
\usepackage{makeidx}
\usepackage{graphicx}
\usepackage{array}
\usepackage[table,xcdraw]{xcolor}
\usepackage{pgf-pie}
\usepackage{framed}
\usepackage{amssymb} 
\usepackage{enumitem}
\usetikzlibrary{calc}




\makeindex
\graphicspath{{images/}}
\date{}
% Title Page

\begin{document}
\begin{titlepage}
	\centering
	\newcommand{\HRule}{\rule{\linewidth}{0.5mm}}
	\textsc{\LARGE Goldsmiths College}\\
	\textsc{University of London}\\[1.5cm]
	\textsc{\Large BSc (Hons) Computer Science}\\
	\textsc{Department of Computing}\\[0.5cm] 
	
	
	\HRule \\[0.5cm]
	{ \huge \bfseries Can virtual reality headsets aid in creating a more realistic facial composite?}\\[0.4cm] % Title of your document
	\HRule \\[1.5cm]
	
	\begin{minipage}{0.4\textwidth}
		\begin{flushleft} \large
			\emph{Author:}\\
			Angela \textsc{Pham} % Your name
		\end{flushleft}
	\end{minipage}
	~
	\begin{minipage}{0.4\textwidth}
		\begin{flushright} \large
			\emph{Supervisor:} \\
			Dr. Michael \textsc{Zbyszynski} % Supervisor's Name
		\end{flushright}
	\end{minipage}\\[3.6cm]
\includegraphics[width=2cm]{logo.png}
\end{titlepage}


\renewcommand{\abstractname}{Acknowledgements}
\begin{abstract}
	...
\end{abstract}

\renewcommand{\abstractname}{Abstract}
\begin{abstract}
	...
\end{abstract}


\tableofcontents
\newpage
\listoffigures
\listoftables

\chapter{Introduction}

\section{Background}

Taylor, who pioneered 2D facial reconstruction, noted that forensic art can than be traced back to more than a hundred years\cite{Ref2} and that an early example can be traced from Scotland Yard when a £200 “wanted” poster was created for the capture of Percy Lefroy Mapleton who committed murder of Isaac Frederick Gold. 
\\

The past few decades has seen major technological advancement of virtual reality. Once seen an experience only available to those would have to travel to experience it, much like a cinema, it has now become readily available to most first-world users through means such as smartphones and games consoles.
\\

Many see virtual reality as a means of escapism, but there are some situations in which virtual reality can assist in real-life situations. An example of this is within the healthcare system. Lange et al  * conducted studies on elderly participants and found that their motor capacity could be improved through the use of technology.
\\
 
This paper looks to discuss whether or not virtual reality can help within the law enforcement setting when generating facial composites.

\section{Motivation}
The past few decades has seen major technological advancement of virtual reality. Once seen an experience only available to those would have to travel to experience it, much like a cinema, it has now become readily available to most first-world users through means of smartphones, games consoles etc. With most first-world users having access to VR, they can be exposed to experiences that they have never been exposed to. These new experiences allow the possibility or users to have a different perception to what they initially had before.  
\\

With this is mind, the main motivation is to develop a series of experiments to see is there is clear correlation between VR and the user’s emotion. The difference between this project and documentaries like \textit{Clouds over Sidra}, is that the project will be developed with the user in mind. This means that the user will be more immersed on the virtual environment and the main focus will be the user and not the narrative. The experiments will have the user embody another mammal, a chimpanzee. The reasoning behind using a chimpanzee is to see if they feel any different emotions to being someone else other than human and to see if there are any differences. 

\section{Aims and Objectives}
For this project, our aim is to examine the users’ initial feelings towards animals that are held captive. We will then extrapolate these results and put them into a clear and concise format. We will then conduct a series of experiments that take advantage of use of VR and test them on the same user. These scenarios will be programmed in Unity using C\#. The results gathered from the experiment will be compared to see there was a clear correlation between the users’ emotion when exposed to virtual reality, in comparison to the user’s initial feelings. 
\\

In order to accomplish this aim, the following objective must be fulfilled:
\\

•Review existing studies on VR experiments, regarding embodiment and emotion
•	Identify the approaches needed to create an immersive environment
•	Code a scene that uses VR and several computing techniques to make a user embody an animal within the virtual environment
•	Investigate user’s perceptions towards animals before the investigation, and after the investigation
•	Compare and identify if there are any clear outcomes once the investigation has been conducted

\section{Contribution}

By the end of this project, the theoretical understanding of virtual reality should be known:
•	A way to develop a virtual environment to induce emotion
•	Techniques to “embody” a character in VR

\section{Thesis Organisation}
The structure of this thesis is as follows:
\\

\textbf{\textit{Chapter 2: Literature Review}}
\\

In this chapter, discussions the past current methods use to generate facial composites and what the key requirements are needed to create an effective piece of software.
After that we discuss the role that virtual reality currently has within the police field and the future potential. 
There is also emphasis on what makes a good facial composite will be reviewed and how VR headsets can improve the current systems implemented. 
\\

\textbf{\textit{Chapter 3: Research Methodology}}
\\

This chapter outlines the current scope of the research and what research approach and method was used in this paper
\\

\textbf{\textit{Chapter 4: Design, Implementation and Testing}}
\\

For this chapter, we take what we have learnt and implement them practically. Chapter 4 includes a description of what each scenario should comprise of. There is also a discussion on several types of current software and the potential they have in creating an effective 3D facial composite. For our investigation, the use of multiple types of software and hardware will be used.  This chapter also discusses any justification behind using certain things within the project.
\\

\textbf{\textit{Chapter 5: Results}}
\\

Chapter 5 discusses the final version of the investigation and how it was conducted. Research will then be conducted after the experiment is over. All of the results and findings are then compared and summarised within this chapter. 
\\

\textbf{\textit{Chapter 6: Discussion}}
\\

This chapter includes an in-depth discussion on the what the results mean regarding virtual reality and facial composites. Strengths and weakness of the experiment carried out are also discussed.
\\

\textbf{\textit{Chapter 7: Conclusion}}
\\

For this last chapter, we evaluate the experiment conducted and what can be done in the future to further improve the results and what problems now arise from the research and can be investigated.

\chapter{Literature Review}
\section{VR Embodiment}
The concept of embodiment covers several disciplines in the scientific field. Regarding neuroscience, embodiment describes the self-perception of your own body. Our bodies are essentially a display device for our mind, and stimulates our senses when interacting with its environment. Biocca states that the body is “fundamental communication hardware” that is a “simulator for the mind” [1].  In virtual reality, the body is used far more than that of a game controller.
\\

Embodiment and virtual reality are symbiotic. With regard to virtual reality, embodiment is a term highly associated with how a user becomes involved with the technology. On the other hand embodiment, can be graded by the level of immersion felt by the user; the greater the immersion the greater the experience. The main purpose of virtual reality is to replicate real life virtually, and to proceed with actions we are unable to do i.e. fly, explore, re-live the past, etc. 
Currently virtual reality is achieved with head mounted displays (HMD), more commonly referred to as virtual reality headsets

\section{Past Experiments}
\subsection{Toy Box}
There are many past, current and future works that have been developed or started that provide a standard of immersion that invokes emotion. Emotions such as happiness, empathy, anger, wonder, and sadness can be felt through a library/store packed with applications and videos.  A few users have also admitted feelings of realisation of time that has passed since childhood. John Patrick Pullen, while attending Consumer Electronics Show (CES) in 2016, described his feelings towards a playing the application Toy Box. The application replicates the imaginary stage of playing toys with the effects coming to life – for example a shrink gun would shrink the target and change elements like voice changes to encapsulate the user. The level of immersion is increased when an action produces an intended action or change in scenario. The ability to have another participant join in also add to the primary user’s sense of immersion and perspective. These attributes combined, made Pullen feel nostalgic and stated that the whole experience brought “tears to his eyes”.

\subsection{Black Pain}
In alignment with the current problem our society faces with race, simulations have gone as far as to demonstrate the animosity felt towards the African American community. The Virtual Reality Company created Black Pain as a means of creating empathy. Guy Primus, CEO of The Virtual Reality Company stated “Every Medium has its own strength and weaknesses, and a master storyteller in any medium can build empathy. In VR, however, your senses are engaged in a way that blurs the line between fantasy and reality”. In terms of how virtual reality provides immersion Primus continued “When you have the ability to look into someone’s eyes and have them look back into yours, things get more real. You can’t look away. Everything is happening in real-time, and that creates a different level of emotion. Those emotions can be positive or negative, but they are very real.”
\\
  
Black Pain puts you in the shoes of an African American teenager, and being the centre of racial discrimination and abuse by forces and people deemed to be racist from recent racial events – most recently and notably, the police forces in America. Many critics ask whether or not virtual reality can force empathy or agitate those who feel like the application may be insensitive. 

\subsection{Rubber Hand}
Rubber hand is an illusion created for the purpose to trick users into believing the prop hand that is touched can be felt on the physical hand of the subject. This idea was replicated into virtual reality form by Ye Yuan, to prove that immersive VR, can have the same effect as reality. What this also proves is the potential of virtual reality coming close to reality. Ye Yuan could prove this theory by testing the visual feedback and responsiveness of the subject. Participants would be requested to use virtual reality headsets, and the application be modified to perfectly resemble the location of the hand in the virtual environment. This would require every part of the project to be synchronous with each other to be provide the most accurate results.
What Ye Yuan was able to prove through virtual reality is the effects of impact and threat on perceived ownership of a limb. In addition, the project shows that distortion although a common fault of virtual reality, did not impact their results. The dimensions of the arm were integral to user feeling immersed, and that animation also played a large role in furthering VR’s immersive ability.
 \begin{itemize}
	
	\item ADD more
\end{itemize}
\chapter{Research Methodology}
Within this chapter, discussions about the approach to gathering first hand research are made. The process of gathering information is also disclosed. Results that are gathered our strategy are then collected and analysed to see if there are any popular opinions or not. Along with the research gathered from the literature review, a clear set of requirements are distinguished. We will also explore any ethical considerations and the research limitations of the project.

\section{Research Method and Approach}
Regarding collecting data about users and Virtual Reality, a basic research method was conducted in our quantitative research approach. Gathering general information through a larger amount of people was the best solution to gain clear understanding. This allows us to draw up statistics and observe if there are distinct trends. To collect this data, options such as survey and questionnaires can be conducted.  
\\

In contrast to basic research, which looks at expanding knowledge, we felt the project could also benefit from applied research. Based on results gathered from the quantitative method, a clear and more detailed set of questions can be formed. Asking the reasoning behind such answers are also beneficial to our project because we need to form a justification towards a user’s emotions. Emotions can difficult to measure and study, due to every individual experiencing feelings a certain way. So, to gather research about a user’s emotions asking questions such and why are beneficial. For gathering qualitative research, we decided to conduct an interview. Another beneficial way of gathering qualitative research would have been an observation of the user regarding VR. 

\section{Ethical Considerations}
Before conducting all of research, written consent letters were given to each of the subjects and told them that the participation was voluntary. This was to reassure participants that they had no obligated to participate. Subjects were also told that all their details would be treated in confidence and that there would be no identifiable information retained.

\section{Data Collection Method and Tools}
For this research, questionnaires and interviews were conducted. An advantage of conducting questionnaires is that there were more people willing to participate. Also with the choice of completing the questionnaire online, there would be less bias in comparison to someone being questioned and interrogated. However, on the other hand, completing the questionnaire online means that many users neglect to answer questions and may discuss answers between acquaintances, which could lead to bias.  
\\

To conduct a questionnaire, a series of questions were made online. These interviews were in depth and relied on the participants to be able to express their emotion and opinion as to why they felt a certain way towards something. This interview was beneficial as it meant that every question was more likely to be given an answer in comparison to questionnaires handed out online.
\\

The tools used to gather data from the interview was through written notes. The option to have an audio recording was available but the participant felt uncomfortable, so it was stopped.

\section{Quantitative Method - Questionnaire Process} 
The questionnaire process started in January 2017, in which a questionnaire was made on online. The questionnaire comprised of several questions related to VR and the user’s opinion and experiences. The questionnaire was handed out electronically to several people through a web link. The questions were as follows:
 
 

\section{Quantitative Method – Questionnaire Findings}
A total of eleven people participated in the questionnaire given out. When analysing the results, there were a few key findings to consider when creating the experiment. For results on all questions and bar charts, please refer to appendix



	
Looking at Figure \ref{graph1}, it is evident that more than 50\% of users had their emotions affected from the use of virtual reality. A key note to suggest when conducting further surveys or interviews is to ask what emotion the participant felt i.e. rage, happiness etc. Another note to consider is the reasoning behind their emotion. Was it because of the storyline? Or possibly the graphics? Or even because of what hardware was used? These factors could be seen a positive or negative. In the example of hardware, they may have felt joy from using a controller that allowed them to be more immersed within the virtual environment. But on the other hand, they may have felt anger and frustration from the head-mount display being too heavy for them. This could affect results because situation in which they felt anger from an external cause, this doesn’t necessarily mean that the virtual environment had affected their emotion. Another point is the user’s mindset and emotion before entering the virtual world.  



A key finding from this result here is that there is a 50\% divide between the positive and the negative thoughts behind using hands in the environment. The 45.45\% that noted “Other” wrote “n/a” so essentially becomes void. With the participants having a 50\% divide on if they like having hands in a virtual environment, we should consider adding it to our experiment as there is a potential for it to become more immersive. This in turn could perhaps make the user feel embodiment as they can “view” themselves.


 
 The results gathered from this question shows a clear preference of users wanting to move around the virtual environment themselves. This shows that participants prefer being within the world and interacting, as opposed to just watching the world around them, much like in 360${}^\circ$ video. The question of why would be an interesting topic to discuss with the interviewee if they also agree. 
 
\section{Qualitative Method – Interview} 
The interview process with the participant occurred in the middle of February. The interview process consisted of myself and one participant. The discussion took place in a room that provided an Oculus Rift, Touch controllers, a computer monitor, a computer able to handle the minimum requirements of the Oculus, a table and an office chair. For the interview process, the participant was asked if they had any experience in using VR, to which they affirmed. They were then shown a 360${}^\circ$ video on the screen once and once using the Oculus Rift. Once both videos were shown they were asked several questions. The interview notes are available  to view in Appendix 
 \begin{itemize}
	
	\item ADD interview questions to appendix
\end{itemize} 

\section{Qualitative Method – Interview Analysis}
For ease, the participant of the interview is called Subject.
When watching the first short film on the monitor, through YouTube, Subject mentioned that it would “be better” watching on the Oculus Rift. When asked about any emotions they felt, Subject said “I felt sadness towards the story, but I wasn’t invested”. When asked the reason why, Subject mentioned “the story of getting older was upsetting”. Subject noted that because they were moving the mouse around a lot, they were missing out on the story. Subject also noted that they didn’t feel immersed within the story, with the reasoning being there was no interaction. This strengthens the point that for an environment to be immersive and the user to be more emotionally attached, the user must be the one to interact.  

When watching the second film, which was an adventure 360${}^\circ$ film, Subject was asked on their opinions. The fact that it was easier to look around compared to using a mouse was mentioned. But again, Subject felt like they were too busy looking around to see what was going on. When talking about their experiences in using VR they mentioned that they liked playing games to movies because they should move and look a certain way, much like in real life to create the storyline. Whereas in 360${}^\circ$ video the storyline happens regardless of where you look. Subject noted 360${}^\circ$ video to being like a “fly on the wall”. 

Subject was also asked when playing their favourite VR game, what their favourite things about it were. Storyline, characters looking at Subject, open-world were noted as great reasons why they play. When further asking about the game, A mentioned that within the game you must play as a robot that can teleport, the game is called Robo Recall. When asked if they felt like a robot when playing they mentioned “sort-of”. Subject mentioned that using the controllers at an awkward position throughout the whole of the game was the reason as to why no controllers would be a better alternative. Subject mentioned that they were heavy to hold and would have to take frequent breaks in between to stop their arms getting sore. The term “Gorilla Arm” may possibly be correct term as to what Subject was mentioning. When conducted the experiment, seeing if the lack of physical controller could prove beneficial towards embodiment and remedy “gorilla arm”.

\section{Research Conclusions}
The research conducted in the brief time, proved to be extremely helpful in attaining a grasp on what the experiment should require to sustain comfortability.
 \begin{itemize}
	
	\item ADD more
\end{itemize}

\section{Research Limitation}
When conducting the research for the project, upon analysis there were several limitations. Firstly, the number of participants that had filled in the questionnaire was relatively small at 11 participants. A longer amount of time to complete the project would have provided potential subjects a chance to complete a questionnaire.  For the qualitative research, the use of a observational study would have been helpful.  Also, the participant for the interview should have had more interaction than just 360${}^\circ$ video and prior virtual reality knowledge. Moreover, more participants thoughts on what embodiment encapsulates would have been preferable. 
\chapter{Design}

\section{Scenario Overview}

\subsection{First Experiment}
For the first experiment, the user will be in an open space. They will assume the body of a chimpanzee within the environment. The user will be able to move around using gestures with the aid of the Leap motion. Once the users feel comfortable navigating around the scene. The scene will then stop.  

\subsection{Second Experiment}
For the second experiment, the user will be placed in a jungle scene. They again will assume the character of the chimpanzee. However, there will be a AI agent that will attempt to chase them down the road. Other chimpanzees can be seen and are also attempting to run away. A loud gunshot is heard and then the screen fades to black.

\subsection{Third Experiment}

For the final experiment, the user is placed within a metal cage. Humans will walk around staring and pointing at the user. It will initially start off with one human and will gradually build up.   


\section{Hardware Requirements}
\subsection{Display}
When discussing different variants of virtual reality systems, the types of displays used tend to fall into different classes. These classes are dependent on the level of immersion or presence that the user feels, when using the system. Costello  classed each of these various implementations into three distinct categories; non-immersive, semi-immersive and fully immersive. These levels of immersion presence are dependent on several parameters which were noted as “level of interactivity, image complexity stereoscopic view, field of regard and the update rate of the display”. 
\\

The least immersive virtual technique used, would only just require a desktop. This would be ideal in a situation that doesn’t require a high level of graphics or specialist hardware and when cost must be kept to a minimum. Semi-immersive systems tend to use a larger display to increase the users sense of immersion. An example of this would be a flight simulator, which would consist of a large display and chair but no other types of specialist hardware.  Fully immersive VR can be seen in environments like the CAVE (Cace Automatic Virtual Environment). These environments typically consist of multiple projections on the floors and walls, speakers, tracking sensors, sound, video and some type of visual headset.  One benefit for use of a CAVE is that multiple people can experience it at the same time. Also, another benefit to using the CAVE, is that there is less probability of experiencing simulator sickness due to the user being able to see their own body. However, for our user to have the sense of embodiment they must see themselves through the chimpanzee’s standpoint. Also in the interest of our investigation, it is not a requirement for multiple users to experience the same thing at the same time.
\\

Making an immersive environment means that the restriction of body movement shouldn’t be kept to a minimum. However, in the interest of our investigation, limiting movement may evoke more emotions, allowing the user to relate to the chimpanzee in our scenario. Also, for our user to have the sense of embodiment they must see themselves through the chimpanzee’s standpoint. Therefore, the use of a head-mounted display (HMD) would be ideal in the case of our experiment. 
\\

Melzer from Kaiser Electro-Optics Inc. called HMDs “personal information-viewing devices” . HMDs work by displaying two separate images in front of each eye, making the images appear 3D. Popular HMDs include the HTC Vive, Oculus Rift, Samsung Gear VR and much more. 

A 3D display system means that, for our experiment, we can create an environment of our own choice.  Using a three-dimensional display is also more beneficial compared to a two-dimensional display due to the fact sense of depth is created. Creating a perception of depth means that our brains are more likely to understand objects and could possibly allow the user to embody the avatar more. 
\\

After consideration, the Oculus Rift was seen to be the best device to move ahead with the experiment. The Oculus is much lighter than the HTC Vive, therefore making it more comfortable for the user. Also, the initial setup requirements are easier on the Rift compared to the Vive. This is because the Vive requires fixed base stations to pick up real time movement of the user. This means that the Vive is only playable in areas that have the ‘Lighthouse’ base stations. Another note to mention it that users who have never used the Vive may be prone to accidents, do to wiring. Also, having already having access to the Rift and experience made it the clear choice. 
\\

For the purpose of the investigation, the hardware specifications in order to use the Oculus Rift are shown in Table \ref{hard-spec} page on  \pageref{hard-spec}.

\begin{table}[]
	 \centering
	\begin{tabular}{
			>{\columncolor[HTML]{C0C0C0}}l l}
		\multicolumn{2}{c}{\cellcolor[HTML]{C0C0C0}\textbf{Hardware Specifications}}                                                                                                                                   \\ \hline
		\multicolumn{1}{|l|}{\cellcolor[HTML]{C0C0C0}\textbf{Graphics Card}}    & \multicolumn{1}{l|}{\textit{Titan X}}                                                                                                \\ \hline
		\multicolumn{1}{|l|}{\cellcolor[HTML]{C0C0C0}\textbf{CPU}}              & \multicolumn{1}{l|}{\textit{Intel Core i7-6700K}}                                                                                    \\ \hline
		\multicolumn{1}{|l|}{\cellcolor[HTML]{C0C0C0}\textbf{Memory}}           & \multicolumn{1}{l|}{\textit{16GB}}                                                                                                   \\ \hline
		\multicolumn{1}{|l|}{\cellcolor[HTML]{C0C0C0}\textbf{Video Output}}     & \multicolumn{1}{l|}{\textit{HMDI}}                                                                                                   \\ \hline
		\multicolumn{1}{|l|}{\cellcolor[HTML]{C0C0C0}\textbf{USB Ports}}        & \multicolumn{1}{l|}{\textit{\begin{tabular}[c]{@{}l@{}}3 x USB 3.0 ports (One for each sensor x2, \\ one for the HMD)\end{tabular}}} \\ \hline
		\multicolumn{1}{|l|}{\cellcolor[HTML]{C0C0C0}\textbf{Operating System}} & \multicolumn{1}{l|}{\textit{Windows 10}}                                                                                                                                              \\ \hline
	\end{tabular}
\caption{Hardware Specifications for Oculus Rift}
\label{hard-spec}
\end{table}

\subsection{Input Device}

Leap Motion was initially a device that allows users to interact with their computer desktop, through the use of hand gestures in the air. The device detects hand motions and using its sensor. In early 2016, software was developed for the Leap Motion to allow the device to track hand motions whilst attached to VR headset. The Leap Motion will be used in the experiment to track hand movement and a possibly way to provide locomotion. When using this device, a USB port is required along with an USB extension cable. 

\section{Software Requirements}
To conduct our experiment, the following software was required and noted below

\subsection{Unity}
Unity is a game engine that allows users to develop games and simulations. For our experiment will use unity to create our environments for each of the experiments. For each of the scripts, the were written in C\#. This also meant that we could import assets from the Asset Store such as assets that allow us to create realistic scenarios. We used the personal plan of Unity and version 5.6.1. Unity requires an operating system of Windows 7 or higher.

\subsection{3DS Max}
This software is used in conjunction with unity to allow us to manipulate the rigging on a pre-rigged model to fit our chosen method of locomotion for the avatar. The model, best suited for the experiment, was that of a chimpanzee and was purchased online (see Assets). 3DS Max was used due the fact that only the. fbx file was pre-rigged. 3DS Max require Windows 7 or later.

\subsection{Leap Motion Orion}
This software allows us to use the Leap Motion device in combination with a VR headset. Orion is able to track the user’s hands gesture and motions faster than the Leap Motion Desktop controller software.  

\subsection{Mixamo and Mecanim}
To animate the other chimpanzees within the experiment, Mixamo is required in conjunction with Unity’s Mecanim system to animated the pre-rigged chimpanzee. This also allowed for smoother transitions in the experiment between each of the states.

\section{Locomotion Research}

Locomotion describes the act of moving from one place to another. Due to the different scenarios, there are several methods of locomotion, each with their own benefits and drawbacks. For each method, we will give a brief description along with what scenario they are ideal with. A discussion at the end will disclose what method would be ideal for our experiment along with our justification as to why. 


\subsection{Traditional artificial locomotion}
One of the most common ways to move around within virtual realms is to use a physical device that has buttons which, when pressed or moved, controls the movement of a character. Examples of devices include keys on a keyboard or an analog stick on a controller. A benefit of using a directional controller is that users have a familiarity with such devices, making them a common solution to locomotion. However, there has been some known instances where users have felt motion sickness when using a physical controller . There have been many theories as to the reason why users feel motion sickness. Mason states “A widely accepted theory into the reason why motion sickness is caused, is due to the mismatch between the senses that keep your balance” . Within VR, the same motion sickness is felt because of the sensory mismatch. For example, in the real world, some people feel motion sickness in a vehicle. This is because when a subject is stationary in a vehicle, the subject’s peripherals can sense the motion of the car, hence causing a sensory mismatch. This type of method is ideal in most scenarios, but reduces the amount of embodiment.

\subsection{Real-world movement: Virtual world movement} 
Another way that is one of the most natural ways to move around in a space is to have a 1:1 scale of the real world to the virtual world. This takes advantage of the sensors tracking the user and moving the avatar when precisely to the position where the user is. This creates a high level of immersion, but is limited to the scale of the room. This also means that users don’t have a need to use controllers. Scenarios where there is not much movement required are ideal. The company Re’flekt in Munich were able to develop a concept called “The Walking Gain” to create a new retail experience for Audi, which allows people to view their customised cars before they were physically made . “The Walking Gain” worked by taking the users movement vector and multiplying it by a factor shift of two in the virtual tracking space across two floor axes. This allowed people to move around a bigger area surface compared to their limited real-world space, and still kneel down without the shift.  

\subsection{Teleportation}
One of the most popular methods of locomotion in virtual reality that a lot of developers use is teleportation. This allows the user to point to the area that wish to go and when a button is triggered, the user is then warped to the chosen area. Examples of applications that use this method are the games Robo Recall , Damaged Care and Vivecraft . The key advantage with this method of locomotion is that vast amounts of space can be covered with the click of a button. On the other hand, this reduces the amount of immersion and embodiment as it if far more relaxing to point and click, which require less physical effort. Sandbox environments would benefit from this type of locomotion.

\subsection{Omnidirectional treadmills}
Omnidirectional treadmills (ODT) refer to a mechanical device that allows the user to move around freely in any direction without the virtual field with their legs. An advantage of using this method is that because of the use of a treadmill, users can walk and run around multiple directions all whilst in staying in the same space. This also allows for full-body immersion as the user isn’t having to rely on a handheld console for movement. However, a major drawback to using omnidirectional treadmills is mainly cost. Examples of ODT’s include Virtuix Omni , Infinadeck  and Cyberith Virtualizer , costing a minimum of £500 or higher. Many users who tested ODT’s agreed and noted that using such device required adjusting their normal walking motions on the base. Oliver Kreylos also stated that using a ODT, specifically the Cyberith Virtualizer, is “not like natural walking”  . Shanklin described his experience on the Virtuix Omni similar to that of “moonwalking ” whilst wearing “slippery bowling shoes”. So, whilst ODT’s do allow full body immersion, there is a possible chance that users will have a learning curve when using this particular device.   

\subsection{Gaze Navigation}
Another method that provides movement virtually is gaze navigation. Gaze navigation works by looking at a certain point within the virtual world which triggers movement towards that direction. This method is beneficial towards immersion as it doesn’t require a handheld controller. On the other hand, a disadvantage is that it may require more technique and precision from a user to move in there intended direction.  Environments that aren’t complex and don’t require frequent head movements are ideal in this situation, as this can cause motion sickness.  

\subsection{Motion sensor movement}
This locomotion method requires the user to move the part of their physical body to simulate movement. Examples of this are ArmSwinger VR  . This works by a user physically swinging their arms to simulate the motion of their arms when walking in the real world. This is a valuable method when creating a more physically immersive environment. The only limitations that this type of method is that it may be strenuous and physically demanding of a user, which means the environment shouldn’t be too vast for it to be appreciated. Another developer, Huge Robot, also created a method called the CAOTS system  which tracks head and arm motion to calculate the pace and direction in which a user wants to move.

\subsection{Other experimental methods}
There are also more experimental methods to locomotion that aren’t as frequently used. Examples of this include astral body which require switches between first and third person controllers. Another example is room pacing which requires allows a user to continue walking an arbitrary distance in a given direction in the virtual environment, whilst pacing around in a physical play area. One developer that created this type of method is Tekton Games, who created the WalkAbout  locomotion system. Games such as Lone Echo  makes use of controllers to grab and pull the user close to or further away from their environment, to move around space. 

\subsection{Locomotion research conclusion}
Based on the research of locomotion in VR, a key factor to consider is how immersive each method is. Using a traditional gamepad controller will not allow the user to feel that they are embodying an chimpanzee. This is because it is not realistic and this method is more geared towards gameplay rather than an experience. These are the same reasons why teleportation and gaze tracking would not be ideal in experiment. Real-world movement would be an acceptable option, but due to the HMD chosen, the Oculus Rift, the tracking would not be as fast or as accurate as the HTC Vive. Another thing to consider is that the Oculus Rift isn’t really designed with walking in mind, a majority of applications are standing or sitting. ODT’s are also an acceptable option, but with limited resources it would not be viable. Motion sensor tracking when used in conjunction with a device such as the Leap Motion may be a plausible solution to locomotion within the experiment, to allow the user to move. It may also allow the user to feel embodiment, due to it feeling the realistic physically.

 
\section{Final Design}
\subsection{First Experiment}
\subsection{Second Experiment}
\subsection{Third Experiment}
\section{Script}
\subsection{DistanceTravelled}
\subsection{Locomotion}
\subsection{Animation}


\chapter{Results}
\chapter{Discussion}
\chapter{Conclusion}
\chapter{Evaluation} 




\addcontentsline{toc}{chapter*}{References}
\renewcommand\bibname{References}
\bibliographystyle{unsrt}
\bibliography{references}


\chapter*{Bibliography}
To be added..

\chapter*{Appendix}
To be added..



\end{document}          
